% !TeX root = ../main.tex

\ustcsetup{
  keywords  = {情绪分析, 大语言模型, 心理健康},
  keywords* = {Emotion Analysis, Large Language Model, Mental Health},
}

\begin{abstract}
近年来,心理健康问题日益受到关注,智能对话系统逐渐成为辅助心理健康服务的重要工具。传统的心理健康干预依赖专业心理咨询师,但由于资源有限、成本高昂,许多有需求的个体难以获得有效帮助。基于大语言模型的智能对话系统为此提供了潜在解决方案。然而,现有的情感对话模型在情感理解与共情能力方面仍存在不足,难以充分满足心理健康领域的需求。为提升心理健康智能服务的精准性、自然性与共情能力,本文设计了一种面向心理健康与语音交互场景的情绪分析大模型。本文的主要贡献如下:

1. 本文设计了一种面向心理健康场景的情感对话大模型 PsycoLLM。现有对话系统往往缺乏对心理健康场景下情感变化的深入理解,难以提供高质量的共情交互体验。为此,本文构建了高质量的心理健康对话数据集,涵盖单轮问答、多轮对话及知识驱动对话,并采用有监督微调方法优化模型,使其在心理咨询任务中具备更强的情感共鸣与问题应对能力。此外,本文设计了一套心理学评估测试集,以量化评估模型在不同心理咨询任务中的表现。实验结果表明,模型在心理健康对话场景中的客观评估准确率较基线模型提升 10\%,并在多个心理学指标上表现更优,展现出更强的情感理解与交互能力。

2. 本文设计了一种多模态情感描述大模型 SEMO-LLM,以克服现有语音情绪分析方法主要依赖单一标签分类、难以捕捉情绪变化及连续性表达的局限性。该模型融合语音声学特征与文本信息,生成更为细腻和动态的情感描述文本。模型采用稀疏桥接 Transformer 结构对齐语音特征与文本特征,并借助大语言模型生成符合语境的情感描述。在基于 IEMOCAP 和 MELD 构建的语音情感描述数据集上的实验结果表明,模型在多个文本描述评估指标上均优于基线模型,为语音情绪分析提供了一种新型的大模型解决方案。

3. 本文基于 PsycoLLM 和 SEMO-LLM 设计并实现了一套大模型驱动的心理健康与情绪分析系统,以进一步提升心理健康智能服务的落地应用效果。该系统集成了心理健康对话支持与语音情感分析功能,提供更智能化、个性化的心理健康服务。在心理健康对话支持方面,系统基于 PsycoLLM 提供更具共情能力的交互体验,使用户能够获得更自然的心理支持反馈;在语音情感分析方面,系统利用 SEMO-LLM 解析语音输入,生成情感描述,辅助心理健康评估,并为个性化情绪干预提供有效支持。


\end{abstract}

\begin{abstract*}
In recent years, mental health issues have received increasing attention, and intelligent dialogue systems are becoming essential tools in supporting mental health services. Traditional mental health interventions rely on professional therapists; however, due to limited resources and high costs, many individuals in need struggle to access effective assistance. Large language model (LLM)-based intelligent dialogue systems offer a potential solution to this challenge. However, existing emotional dialogue models still exhibit deficiencies in emotional understanding and empathy, making them insufficient for the demands of mental health applications. To enhance the precision, naturalness, and empathetic capability of intelligent mental health services, this study proposes a large-scale emotion analysis model tailored for mental health and speech interaction scenarios. The main contributions of this study are as follows:

1. This study proposes PsycoLLM, a large-scale emotional dialogue model designed for mental health scenarios. Traditional dialogue systems often lack a deep understanding of emotional dynamics in mental health contexts, making it difficult to provide high-quality empathetic interactions. To address this limitation, we constructed a high-quality mental health dialogue dataset, covering single-turn question and answer, multi-turn dialogues, and knowledge-driven conversations. Additionally, the model was optimized through supervised fine-tuning to enhance its emotional resonance and problem-solving capabilities in mental health counseling tasks. Furthermore, a psychological assessment test set was developed to quantitatively evaluate model’s performance across different counseling tasks. Experimental results show that PsycoLLM improves objective evaluation accuracy by 10\% compared to baseline models in mental health dialogue scenarios and demonstrates superior performance across multiple psychological indicators, showcasing enhanced emotional understanding and interaction capabilities.

2. Existing speech emotion analysis methods primarily rely on single-label classification, limiting their ability to capture emotional variations and continuous expressions. To address this issue, this study introduces SEMO-LLM, a multimodal large-scale emotion description model that integrates acoustic speech features and textual information to generate more nuanced and dynamic emotional descriptions. The model employs a sparse-bridging Transformer architecture to align speech and textual features, leveraging LLMs to generate contextually appropriate emotional descriptions. Experimental results on speech emotion description datasets constructed from IEMOCAP and MELD demonstrate that model outperforms baseline models across multiple text description evaluation metrics, offering a novel LLM-based solution for speech emotion analysis.

3. To further enhance the practical application of intelligent mental health services, this study designs and implements an LLM-driven mental health and emotion analysis system based on PsycoLLM and SEMO-LLM. This system integrates mental health dialogue support and speech emotion analysis functionalities, providing more intelligent and personalized mental health services. For mental health dialogue support, the system leverages PsycoLLM to enable more empathetic interactions, allowing users to receive natural and supportive psychological feedback. In speech emotion analysis, the system utilizes SEMO-LLM to process speech inputs, generate emotional descriptions, assist in mental health assessments, and provide effective support for personalized emotional interventions.

\end{abstract*}
