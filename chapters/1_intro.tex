% !TeX root = ../main.tex

\chapter{绪论}

\section{研究背景及意义}

随着社会节奏的加快与个体所承受压力的持续增长,心理健康问题逐渐成为全球关注的重大公共卫生挑战。心理健康问题对个人、家庭和社会的影响正逐年加重,特别是焦虑、抑郁等情绪问题的普遍存在引发了广泛关注\cite{陈_李_2022}。传统的心理健康对话支持手段依赖于面对面咨询,然而,受到资源匮乏、时间限制和地域分布等因素的影响,这种方式难以满足日益增长的心理健康需求。因此,借助人工智能特别是大型语言模型的能力来提供个性化的心理健康对话支持成为一种有效补充手段。

在人工智能技术的快速发展背景下,大型语言模型在自然语言处理、对话生成和情绪识别等领域表现出卓越的理解和生成能力。这些模型通过训练在大规模文本数据上,能够生成符合语境的对话,并具备情绪分析和理解用户情绪状态的潜力\cite{yan_yang_yin_2023}。因此,情绪分析系统逐渐成为心理健康辅助工具中的重要组成部分。情绪分析技术不仅能帮助心理健康从业者更好地评估用户的情绪状态,还能够在智能客服、虚拟助手等智能服务中提供更为个性化和细腻的情绪响应。尽管如此,当前情绪分析在心理健康领域的应用仍面临诸多技术挑战,包括如下部分:

(1)传统单标签分类的局限性

在情绪分析中,传统的单标签分类方法通常只能为每一段文本或语音分配一个固定的情绪标签,例如“高兴”或“愤怒”\cite{Zheng_Liu_Yin_2021}。然而,心理健康对话中的情绪往往具有复杂性和连续性,用户的情绪状态可能在对话中不断变化或交织。例如,用户在表达某种情绪的同时可能夹杂着焦虑或不安,单一标签很难捕捉到这些微妙的情绪动态。这种局限性使得单标签分类方法难以有效支持心理健康场景中的多维情绪理解需求。

(2)语音情绪分析中的多模态特征挑战

语音情绪分析涉及的特征不仅限于文字内容,还包含语音中的音调、音量、语速等声学信号,这些信号常常用于传达情绪的强度和细微变化\cite{罗_冉_杨_豆_2022}。情绪表达的多样性导致不同声学特征组合产生的情绪表现各异,例如,音量增大可能代表愤怒,而语速加快可能代表紧张或兴奋。由于语音信号的多模态特性,模型在处理情绪波动时面临着巨大的挑战。现有的语音情绪分析模型难以全面捕捉这些多模态特征,导致在识别情绪变化时可能存在不足,从而影响模型的准确性和适用性\cite{丁_陈_曾_2023}。

(3)情绪分析数据集的局限与适用性不足

当前情绪分析系统大多依赖于标准化的情绪数据集进行训练,但这些数据集通常缺乏心理健康场景中丰富的情绪细节。心理健康对话中的情绪信息较为复杂,包含细微的情绪层次和用户心理状态的动态变化,然而,现有的数据集无法完全涵盖这些情绪特征。此外,许多数据集并未充分考虑真实场景中的情绪表达方式的多样性,特别是在涉及心理健康的对话中,这些情绪表达可能更为隐蔽且具有特殊性。

\vspace{1em}

在此背景下,本文围绕心理健康对话支持和语音情绪分析两个核心方向,分别设计了 PsycoLLM 和 SEMO-LLM 两个大语言模型驱动的系统。PsycoLLM 旨在针对心理健康对话场景优化大模型的情感理解和共情能力,提升其在心理咨询任务中的应用价值。为此,本文构建了高质量的心理健康对话数据集,并采用有监督微调(SFT)技术对模型进行优化,使其在心理咨询任务中具备更强的情感共鸣和推理能力。此外,为了系统评估模型的实际应用效果,本文设计了心理健康领域的专业评测体系,并通过实验验证了 PsycoLLM 在心理健康对话任务中的有效性。实验结果表明,该模型在多个心理健康任务评估指标上均优于基线模型,展现出较强的适用性和情感交互能力。

此外,本文还设计了多模态情感描述模型,通过结合文本和语音特征,本文设计了 SEMO-LLM 模型,实现对情绪状态的细粒度分析。该模型采用语音处理模块、稀疏桥接 Transformer 以及大模型情感描述生成模块,以增强语音与文本特征的融合能力,提高情感描述的准确性和表达质量。综上,本文的研究不仅为心理健康对话支持提供了基于大模型的解决方案,也为语音情绪分析领域的发展提供了新的技术路径,推动了智能对话系统的进一步优化。

\section{研究现状}

本节对情绪识别和情感生成领域的相关研究进行综述,重点包括对话情绪识别方法、心理健康对话支持方法和语音情绪分析方法的研究现状,并分析各类方法的优势和存在的问题,为本文提出的多模态情感描述大模型方法提供理论背景与研究支撑。

\subsection{对话情绪识别方法研究现状}

对话情绪识别是情感分析的一个分支,旨在结合对话上下文信息和说话者特征,对句子的情绪类别进行判断,并将识别结果应用于后续任务。情绪分类是该领域的重要基础,心理学家提出了不同的分类体系。例如,Ekman\cite{Ekman_1993} 在 1972 年定义了六种基本情绪,包括 anger、disgust、fear、happy、sadness 和 surprise。而 Plutchik\cite{Plutchik_1982} 则将情绪划分为 joy、trust、anticipation、surprise、fear、sadness、disgust 和 anger 八大类别,并指出这些类别之间存在相应的子类。这些分类体系为对话情绪识别的研究提供了理论依据,后续研究者在其基础上进一步扩展应用。

早期的对话情绪识别研究主要依赖于传统机器学习方法。Rozgic 等人\cite{Rozgic_Ananthakrishnan_Saleem_Kumar_Prasad_2012} 通过词袋模型提取文本特征,并结合梅尔频率倒谱系数(MFCC)描述语音特征,同时利用人脸信息提取视频特征。随后,他们采用支持向量机(SVM)进行情绪分类。Jin 等人\cite{Jin_Li_Chen_Wu_2015} 进一步使用线性核 SVM 进行情绪分类,而 Kessous 等人\cite{Kessous_Castellano_Caridakis_2010} 则采用贝叶斯分类器。Zeng 等人\cite{Zeng_Hu_Liu_Fu_Huang_2006} 通过隐马尔可夫模型融合多种模型,从而提升实验效果。此外,Glodek 等人\cite{Glodek_Reuter_Schels_Dietmayer_Schwenker_2013} 采用卡尔曼滤波算法,融合音频和视频信息,以增强情绪识别能力。然而,这类方法高度依赖人工特征工程,泛化能力有限,难以适应复杂的对话场景。相比之下,深度学习模型能够自动学习高维特征,避免了繁琐的人工特征提取,因而逐渐取代传统机器学习方法成为主流。

随着深度学习的快速发展,基于神经网络的模型显著提升了对话情绪识别的表现。Bertero 等人\cite{Bertero_Siddique_Wu_Wan_Chan_Fung_2016} 率先应用卷积神经网络(CNN)进行情绪识别,利用 TextCNN 提取文本特征。然而,该方法未考虑上下文信息,仅适用于独立语句的分类。因此,研究者们开始关注对话中的上下文建模,以提高识别效果。长短期记忆(LSTM)网络\cite{Hochreiter_Schmidhuber_1997} 作为一种能够捕捉时间序列依赖关系的神经网络,被广泛用于对话情绪识别。Poria 等人\cite{Poria_Cambria_Hazarika_Majumder_Zadeh_Morency_2017} 采用基于上下文的 LSTM,并结合注意力机制,以增强对话语境信息的建模能力。Jiao 等人\cite{Jiao_Yang_King_Lyu_2019} 提出了 HiGRU 模型,采用分层 GRU\cite{Cho_van_Merrienboer_Gulcehre_Bahdanau_Bougares_Schwenk_Bengio_2014} 结构,其中低级 GRU 处理单词级特征,高级 GRU 负责捕捉话语级上下文信息。此外,Chen 等人\cite{Chen_Hsu_Kuo_Ting-Hao_Huang_Ku_2018} 设计了多层次循环神经网络\cite{Elman_1990},通过分层建模句子特征及连续语句的依赖关系,以提升情绪识别能力。与此同时,基于自注意力机制的 Transformer\cite{Vaswani_Shazeer_Parmar_Uszkoreit_Jones_Gomez_Kaiser_Polosukhin_2017} 模型以及其衍生的 BERT\cite{Devlin_Chang_Lee_Toutanova_2019}、GPT\cite{Radford_Narasimhan_Salimans_Sutskever} 等预训练语言模型,为情绪识别提供了强大的上下文理解能力,这些模型能够捕捉语言中的隐含情感,使得对话情绪识别的准确性和鲁棒性大幅提升。

近年来,一些创新性的方法进一步推动了对话情绪识别技术的发展。例如,基于注意力-循环机制的模型将BERT模型的强大注意力机制与循环神经网络(RNN)结合,能够更好地理解和跟踪对话中的情感流动,提升了情绪识别的精度,Yang\cite{yan_yang_yin_2023}、Hu\cite{Hu_Bao_Wei_Zhou_Hu_2023} 和 Hu\cite{Hu_Wei_Huai_2021} 等人设计不同级别的编码器和循环神经单元来提取情感特征。然而,双向编码允许当前话语考虑未来的信息,这显然不合逻辑。基于常识的情绪识别方法如 Ghosal\cite{Ghosal_Majumder_Gelbukh_Mihalcea_Poria_2020} 和 Li\cite{Li_Lin_Fu_Wang_2021} 等人则通过引入外部知识库和常识信息,增强了情感分析的语义深度和背景信息,使得模型能够在对话中更全面地考虑情境。然而,这类方法也面临着计算资源消耗大和噪声问题,如何平衡其性能和效率,仍是该领域的挑战之一。此外,基于图神经网络(GNN)的方法,也逐渐在对话情绪识别中显现出潜力。Liang\cite{Liang_Yang_Xu_Huang_Wang_Dong} 和 Shen\cite{Shen_Wu_Yang_Quan_2021} 等人使用 GNN 有效建模对话中的社交关系、说话者之间的情感交互,挖掘复杂的情感传播路径,这些高度结构化的模型倾向于过度拟合特定的数据特征(长对话或复杂的对话关系),这使得它们难以推广到对话内容分布未知的现实场景。

\subsection{心理健康对话支持方法研究现状}

心理健康对话支持主要依赖于传统的心理咨询、心理治疗以及数字化干预手段。这些方法通常涉及专业心理学家的介入,通过面对面或远程的方式进行心理咨询。心理健康服务模式主要包括认知行为疗法、精神分析治疗、人本主义疗法以及正念疗法等\cite{Colby_Ortony_Clore_Collins_1989}\cite{Rogers}。这些方法在长期临床实践中被广泛验证,能够有效缓解焦虑、抑郁等心理障碍。然而,由于专业心理咨询资源有限,尤其是在心理健康需求日益增长的情况下,传统方法在可及性、成本以及时间限制方面存在一定的挑战\cite{Kazdin_Blase_2011}。

随着信息技术的发展,心理健康对话支持逐渐向数字化方向转变。例如,基于计算机和互联网的心理干预成为一种新兴的支持方式。这些干预手段包括基于网络的心理健康教育、移动应用以及虚拟心理咨询。Shen\cite{Shen_Rudzicz_2017} 等人通过网络化的心理健康干预在某些情境下可以获得与传统心理治疗相近的效果。此外,认知行为疗法被广泛应用于数字平台,如 Woebot\cite{Fitzpatrick_Darcy_Vierhile_2017} 和 iCBT\cite{Williams_Andrews_2013}。这些系统采用规则驱动的聊天机器人或交互式课程,为用户提供心理支持,但由于缺乏深度对话能力和个性化反馈,仍然存在局限。

早期的研究利用机器学习算法对心理健康数据进行分析,包括抑郁症预测、焦虑评估以及情感分析\cite{Shen_Rudzicz_2017}。自然语言处理技术也被用于情感识别、心理健康问答系统以及社交媒体心理健康监测\cite{Guntuku_Yaden_Kern_Ungar_Eichstaedt_2017}。然而,这些方法大多依赖于预定义的规则或有限的数据集,无法提供灵活、自然的对话体验。

近年来,大型语言模型(LLM)受到了广泛关注,并取得了显著进展,例如 ChatGPT、PaLM\cite{Chowdhery_Narang_Devlin_Bosma_Mishra_Roberts_Barham_Chung_Sutton_Gehrmann_etal}、Qwen\cite{Bai_Bai_Chu_Cui_Dang_Deng_Fan_Ge_Han_Huang_et} 和 LLama\cite{Touvron_Lavril_Izacard_Martinet_Lachaux_Lacroix_Roziere_Goyal_Hambro_Azhar_et} 等。这些模型在多个任务中展现出卓越的能力,包括问答(QA)、摘要生成和信息提取等。此外,LLM 还被成功应用于多个领域,如医学和法律等\cite{Xiong_Wang_Zhu_Zhao_Liu_Huang_Wang_Shen}。在当前社会背景下,个人与职业需求的增长使得心理健康问题日益突出。由于 LLM 依托大规模数据训练,并包含海量参数,研究表明其具备强大的自然语言理解能力,能够提供合理的反馈。因此,利用 LLM 来辅助心理问题的解决,有望成为缓解心理健康问题的有效策略,使求助者更便捷地获取及时的帮助。

目前,已有诸多研究尝试将 LLM 应用于心理健康领域。例如,Lai\cite{Lai_Shi_Du_Wu_Fu_Dou_Wang_2023} 研究结合预训练的 LLM 与真实专业问答数据,以提供在线咨询服务;SouChat\cite{Chen_Xing_Lin_Zheng_Wang_Liu_Xu_2023} 和 SMILE\cite{Qiu_He_Zhang_Li_Lan_2023} 构建了多轮共情对话数据集,并基于基础 LLM 进行微调。此外,EmoLLM\cite{yang2024emollmmultimodalemotionalunderstanding} 采用类似方法,在心理健康数据集上对开源基础模型进行了微调。尽管这些研究在推动该领域发展方面做出了重要贡献,但仍然存在一定的局限性。首先,当前大多数心理健康微调数据集主要由更大的语言模型(如 ChatGPT)生成。然而,这些数据的生成过程缺乏充分的先验知识和证据支持,可能导致数据集偏离真实的心理交流场景。其次,由于缺乏全面的心理学基准,现有研究的评估方式多依赖专家的主观判断或其他大型模型的辅助评估,这种方式可能难以全面反映模型的实际表现。例如,PsyBench\cite{Zhang_He_Song_He_Zhang_Qiu_Li_Ma_Lan_2023} 采用 GPT-4\cite{Achiam_Adler_Agarwal_Ahmad_Akkaya_Aleman_Almeida_Altenschmidt_Altman_Anadkat_et} 生成的问题来评估模型的性能,但这种评估方式存在一定的局限性。


\subsection{语音情绪分析方法研究现状}

% 语音情绪分析技术,旨在通过分析语音信号中的声学特征,识别说话者的情绪状态。作为情感计算的重要组成部分,语音情绪分析不仅涉及到语音信号处理技术,还涉及到情感计算和模式识别等多学科交叉领域。近年来,随着深度学习技术的快速发展,语音情绪分析的研究取得了显著进展。
语音情感分析技术是通过提取语音信号中的声学参数,实现对人类情感状态的自动检测与分析。作为情感智能研究的关键分支,该领域综合运用了数字信号处理、机器学习以及心理学等多学科知识体系。得益于深度神经网络架构的不断创新,语音情感分析在模型精度和泛化能力方面均获得了突破性提升。

传统研究通常将语音情绪分析视为一个分类任务,即语音情感识别(Speech Emotion Recognition, SER),通过将情感归类至离散的类别(如恐惧、快乐等)来实现对语音情感的识别\cite{El_Ayadi_Kamel_Karray_2011}\cite{Nwe_Foo_De_Silva_2003}\cite{Jiang_Fu_Tao_Lei_Zhao_2019}。近年来,随着新型模型架构的不断发展,该领域的研究取得了显著进展。然而,传统的SER方法存在一定局限性。首先,单一的情感标签往往难以体现情感的细微变化,例如情感的强度和波动。此外,语音情感通常具有多层次和多样化的特性,在一个语音片段中可能同时包含不同的情感状态(如既快乐又紧张)。将语音情感简单地归类到某一类别,可能无法充分捕捉其真实情感状态。同时,由于情感的主观性,不同个体在理解复杂语音情感时可能存在一定的差异,进一步加大了分类任务的挑战性。

鉴于语音情感分类方法的局限性,近年来研究者开始尝试利用自然语言描述语音情感,而非简单的情感标签。受自动音频描述(Automated Audio Captioning, AAC)\cite{Ye_Wang_Yang_Zou_2021}任务发展的启发,SEcap\cite{xu2024secap} 提出了语音情感描述(Speech Emotion Captioning, SEC)任务,该框架包括音频编码器、桥接网络(Bridge-Net)和文本解码器,以自然语言的方式刻画语音情感。

语音情感描述任务的研究主要面临两个核心挑战:一是如何从原始语音输入中提取情感相关的语音特征;二是如何生成高质量、类人化的语音情感描述。针对SEC任务的第一个挑战,即如何从原始语音输入中提取情感相关的语音特征,传统手工特征提取方法(如基于音调、时长、能量、频谱等特征)往往难以全面捕捉语音情感的复杂性,且需要大量领域知识进行特征工程,导致模型的适用性受限。近年来,预训练语音模型(如 Wav2Vec\cite{Schneider_Baevski_Collobert_Auli_2019}、HuBERT\cite{Hsu_Bolte_Tsai_Lakhotia_Salakhutdinov_Mohamed_2021})为语音特征提取提供了新的思路,但直接使用帧级特征仍然计算量较大,且如何进一步提取高质量的情感相关信息仍然是一个挑战。

针对第二个挑战,即如何生成高质量、类人化的语音情感描述,现有研究较少涉及使用自然语言进行语音情感表达。近年来,由于语音情感描述任务涉及跨模态信息映射,即从音频特征到文本的转换,现有模型 WavCaps\cite{Mei_Meng_Liu_Kong_Ko_Zhao_Plumbley_Zou_Wang} 和 SECap\cite{xu2024secap} 在如何有效利用音频特征、提升文本生成质量方面仍存在较大挑战。随着大语言模型(LLMs)的发展,基于预训练大模型的方法为情感描述任务带来了新的机遇,但如何高效地将语音情感信息映射到大语言模型的语义空间,仍然是当前研究的难点之一。

\section{本文主要工作}

本论文来源于国家重点研发计划子课题基于人工智能的健康居住环境测-评-控技术与产品研究(2022YFC3803202)  以及安徽省科技重大专项面向智慧交通的身心协同计算智能传感关键技术研究与应用(202203a05020011)。基于上述研究背景和现状,本研究围绕心理健康对话支持与语音情绪分析两个核心方向,重点探讨基于大语言模型(LLM)的情感对话方法、多模态情感描述方法以及心理与情绪分析系统的设计与实现。本文主要工作主要涵盖以下三个方面:

(1)\textbf{面向心理健康的大模型情感对话方法研究}。为提升心理健康对话系统的情感分析与共情能力,本文提出 PsycoLLM。传统对话系统常面临情感理解不足、共情能力弱等问题,难以有效支持心理健康需求。为此,本文构建高质量心理健康对话数据集,并通过有监督微调(SFT)优化模型,使其具备更强的情感共鸣与问题应对能力。此外,本文设计心理学评估测试集,衡量 PsycoLLM 在不同心理咨询场景下的表现。实验结果表明,经过心理知识注入和优化的 PsycoLLM 在多个心理学评估指标上优于基线模型,展现出更高的情感理解与交互能力。

(2)\textbf{面向语音交互的多模态情感描述大模型方法研究}。为解决传统语音情绪分析方法难以精准刻画情感状态的问题,本文提出 SEMO-LLM。现有语音情绪分析多采用离散分类方法,无法充分表达情感的连续性和复杂性。SEMO-LLM 结合语音处理模块、稀疏桥接 Transformer 及大模型情感描述生成模块,实现语音与文本特征的融合,并生成自然的情感描述文本。

(3)\textbf{大模型心理健康与情绪分析系统软件设计}。系统结合 PsycoLLM 和 SEMO-LLM,包含心理健康对话支持和语音情绪分析功能。心理健康对话支持模块基于 PsycoLLM 提供更具共情能力的交互;语音情感分析模块结合 SEMO-LLM 解析语音数据输入。该系统为智能心理健康对话支持提供了有效方案。

\section{章节安排}

本研究旨在探讨心理健康对话支持和情绪分析方法,进而提出一种基于大模型的多模态情感描述生成方法。研究内容涵盖情绪识别的理论基础、技术方法、数据集构建及其在心理健康、语音交互等实际场景中的应用。通过对现有情感分析技术和大模型的深入研究,本文设计了一种创新的情感分析框架。本文共分五个章节,具体结构安排如下。

第一章绪论:本章主要介绍了研究的背景和意义,阐述了心理健康对话支持和情绪识别技术在自然语言处理和人工智能中的重要性,指出了目前存在的局限性和挑战。重点分析了传统单标签分类方法的不足、多模态情绪分析中的挑战,以及情绪分析数据集在应用中的局限性。

第二章相关理论与技术:本章对心理健康支持和情绪分析的基本理论进行回顾,详细讨论了情感分析中的数据预处理技术,包括文本和语音数据的预处理方法。此外,还介绍了深度学习在情感分析中的应用,包括情感分类技术、深度学习模型及其优化器的选择,为后续章节的模型设计和技术实现提供理论支持。

第三章面向心理健康的大模型情感对话方法研究:本章聚焦于心理健康对话支持领域,探讨了基于大语言模型的情感对话方法。首先介绍了数据集的构建过程,特别是单轮问答、多轮对话和知识性问答数据集的设计和处理。接着,详细阐述了大模型结构设计,包括模型的 Tokenizer、Decoder 模块设计及其微调方法。此外,本章还通过构建心理健康评估基准实验分析了 PsycoLLM 模型在心理健康领域的应用效果。

第四章面向语音交互的多模态情感描述大模型方法研究:本章讨论了面向语音交互的多模态情感描述生成模型。首先介绍了多模态情感描述大模型 SEMO-LLM 的结构设计,重点分析了语音处理模块、稀疏桥接 Transformer 模块及大模型描述生成模块。然后,探讨了情感标签生成、特征提取以及情感描述生成的技术实现,详细描述了情感分析在语音交互系统中的应用。通过实验验证了该方法在语音情感识别中的优越性。

第五章大模型情绪分析系统软件设计:本章介绍了基于大模型情绪分析技术的软件系统设计。包括系统需求分析、系统的整体架构设计和功能模块的实现。系统功能模块设计部分详细讨论了开发环境的选择、系统的主要功能以及如何实现情感分析和情绪识别任务。最后,给出了系统的使用流程和操作指导。

第六章总结与展望:本章总结并分析了本文提出的心理健康对话支持模型以及多模态情感描述模型。并就当前工作还需要进一步改进的方向进行了展望,包括数据集的构建、模型的优化、系统的性能评估等。并对当前工作需要改善和未来的研究方向进行了展望。