% !TeX root = ../main.tex

\begin{acknowledgements}

时光荏苒,转眼间我的硕士学习生涯即将画上句号。回首这段求学之路,心中充满感激之情。在此,我谨向所有给予我关心、支持和帮助的老师、同学、朋友和家人致以最诚挚的谢意。

首先,我要衷心感谢我的导师中国科学技术大学周文罡教授。从论文开题到最终完成,周老师始终以严谨的治学态度和深厚的学术造诣指导着我的研究工作,并在每个关键节点给予精准指导。在论文撰写过程中,周老师总能一针见血地指出核心问题,从整体框架到具体论证都提出了极具建设性的修改意见。周老师不仅以其深厚的学术造诣提升了我的研究水平,更以循循善诱的指导方式培养了我的学术思维。

特别感谢合肥综合性国家科学中心汪萌副院长为我提供的宝贵学习平台和研究机会。在汪院长的支持下,我得以接触到最前沿的科研课题,拓展了学术视野,提升了研究能力。汪院长高瞻远瞩的学术眼光和开放包容的治学理念,让我深刻体会到跨学科研究的重要性。

衷心感谢我的实践导师陈欣老师。在项目实践过程中,陈老师总是毫无保留地分享他的实践经验,手把手地指导我解决各种技术难题。陈老师扎实的专业功底和求真务实的工作作风,让我在实践中快速成长,这些宝贵的经验将使我受益终身。

感谢实验室的各位师兄师姐和同窗好友。特别感谢王飞博士、赵佳琪、刘雨鑫、王锐等同学在实验设计和数据分析中给予的建议;与实验室师兄师弟们一起讨论问题、攻克难关的日子,是我研究生生涯中最珍贵的回忆。还要感谢室友鲍并雷、柴博和周龙飞三年来在生活上的照顾和精神上的支持。

深深感谢我的父母和家人。是家人无私的爱与支持,让我能够心无旁骛地完成学业。每当我遇到挫折时,父母总是我最坚强的后盾;每当我取得进步时,父母又是我最忠实的喝彩者。父母的理解与鼓励,是我不断前进的动力。

最后,向所有在匿名评审中为本论文提出宝贵意见的专家们致以谢意,你们的建议使我的研究更加完善。

研究生阶段的学习不仅让我收获了专业知识,更让我懂得了感恩与责任。在未来的工作和生活中,我将继续秉持"红专并进,理实交融"的校训精神,以优异的成绩回报所有关心和帮助过我的人。

\end{acknowledgements}
